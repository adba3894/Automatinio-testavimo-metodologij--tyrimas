Didėjant WEB aplikacijos apimčiai ir naudotojų kiekiui, tampa vis sunkiau užtikrinti sistemos kokybę rankiniu būdu ir nepriekaištingas kokybės užtikrinimo procesas tampa neatskiriama aplikacijos vystymo dalimi. Tokiu atveju rankinis testavimas atsiranda poreikis naudoti automatinį testavimą.

Atliekant kodo pertvarkymą (\textbf{angl.} refactoring) arba naujų funkcijų įgyveninimą dažnai klaidos aptinkamos jau egzistuojančiuose kodo vienetuose. Tai gali padaryti didžiulę įtaką būsimo produkto kokybei, todėl testavimas turi būti pradedamas jau ankstyvoje programinės įrangos vystymosi ciklo stadijoje. Automatinis testavimas yra efektyvus būdas patikrinti aplikacijos kokybę prieš išleidžiant ją į produkciją.

Šio darbo tikslas yra išsiaiškinti ir pagrįsti kodėl ir kuris automatinio testavimo tipas yra efektyviausias, patikimiausias, bei turintis aukščiausią pakartotinio panaudojimo (\textbf{angl.} reusability) lygį. Bus rašomi testai, naudojant ir išbandant kiekvieną egzistuojantį automatinio testavimo tipą ir samprotaujama, kuris iš jų labiausiai atitinka paminėtus kriterijus. Pavyzdžiam bus naudojama JAVA programavimo kalba ir Selenium bei Cucumber testavimo karkasai.