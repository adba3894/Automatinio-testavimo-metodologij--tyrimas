Tinkamo testavimo karkaso pasirinkimas priklauso nuo projektų, komandos patirties ir turimų išteklių ir turi didelę įtaką internetinių aplikacijų kūrimo ciklo sklandumui bei galutinio produkto kokybei, todėl ši internetinių puslapių kūrimo ciklo dalis yra gyvybiškai svarbi ir negali būti praleista. Pradedant projektą svarbu nusistatyti, į ką WEB aplikacija bus orientuota - ar tai bus aplikacija, kuri valdys didelius įvairių duomenų kiekius, ar aplikacija, kuri turės daug smulkių funkcijų, kurios yra priklausomos vienos nuo kitos ir pagal tai pasirinkti data-driven, behavior-driven ar kitokią automatinio testavimo metodiką, tačiau visgi rekomenduotina automatiniam testavimui naudoti hybrid-driven automatinio testavimo metodologiją, jungiančią daugelio metodikų privalumus į atskirą metodiką, padėsiančią užtikrinti aplikacijos kokybę vertinant ją skirtingais aspektais, kuriuos atstovauja kiekviena iš metodikų.

Tam, kad automatizacijos procesą palaikyti veiksmingą, tiek testuotojai, tiek produkto šeimininkai, vadybininkai, klientai turi norėti ir turėti galimybę atlikti pakeitimus produkte, jei to reikalauja susiklosčiusi situacija.
