Per pastaruosius 30 metų programinės įrangos pramonė padarė didelę pažangą ir pakeitė daugelio žmonių gyvenimą. Šiais laikais skambinant, užvedant automobilį, įjungiant elektrinę viryklę, atsiskaitant debeto kortele, daugelis net nenumanome, kad tokių procesų metu vykdome tam tikrą programinės įrangos kodą. Paprastai programinė įranga susiduria su dvejais iššūkiais: kuo mažesnės programavimo darbų išlaidos ir produkto kokybės užtikrinimas. Kokybiška programinė įranga gali būti apibūdinama kaip naudinga, patikima ir saugi programinė įranga. Programinės įrangos kokybė turi įtakos pasaulio ekonomikai, nacionaliniam saugumui, sveikatos priežiūrai, todėl programinės įrangos kokybė yra didžiulė atsakomybė. Šis darbas yra orientuotas į internetinių aplikacijų kokybę bei turi tikslą atsakyti į klausimą, kaip automatinis testavimas gali prisidėti prie internetinių aplikacijų kokybės.